\documentclass[11pt]{article}
\usepackage[french]{babel}
\usepackage{amsfonts,amstext,graphicx,amsmath,amssymb,ifthen,mathrsfs,multicol}
\usepackage[T1]{fontenc}

\def\ba{\begin{eqnarray}}
\def\ea{\end{eqnarray}}
\def\be{\begin{equation}}
\def\ee{\end{equation}}
\def\O{\mathcal{O}}
\def\H{\mathcal{H}}
\def\hO{\hat{O}}
\def\C{\mathcal{C}}
\def\L{\mathcal{L}}
\def\K{\mathcal{K}}
\def\F{\mathcal{F}}
\def\dd{\left|\partial d\right|}
\def\del{\nabla}
\def\Et{\tilde{E}}
\def\Bt{\tilde{B}}
\def\nn{\nonumber}
\def\cosech{\mathrm{cosech}}
\def\cosec{\mathrm{cosec}}
\def\x{\mathbf{x}}
\def\d{\mathrm{d}}
\def\yt{\tilde{y}}
\def\mn{_{\mu \nu}}
\def\mupn{^\mu_{\, \nu}}
\def\({\left(}
\def\){\right)}
\def\ie{{\it i.e. }}
\def\rpm{r_\pm}
\def\rp{r_+}
\def\rmm{r_-}
\def\rt{\tilde{r}_\pm}
\def\bd{\boxdot}
\pagestyle{empty}

%============================================================
%Macros de configuration pour la mise en forme des s\'eries.
%Marges
\newcommand\mydimensions{\oddsidemargin 0in \evensidemargin 0in \topmargin 0pt \textheight 230mm \textwidth 165mm}
\def\eref#1{(\ref{#1})}
\mydimensions
\parskip = 0.2truecm
\parindent = 0truecm
%Titre avec barre
\newcommand\Header[3]{\phantom{...}
\vskip -2.0truecm \noindent {\obeylines Habib University \hfill Dr. Muntazir Abidi \vskip -2mm \noindent
Dhanani School of Science and Engineering\hfill Integrated Sciences and Mathematics (iSciM) \vskip 1.5mm
\hrule \vskip 0.5truecm}
\begin{center} {\bf\large #1 Exercise Sheet #2} \end{center}
\pagestyle{empty}}
%environnement exo
\newcommand\proclaim[2][\bf]{\medbreak\noindent{#1#2}\enspace\ignorespaces}
\newcounter{exocomp}
\setcounter{exocomp}{1}
\newenvironment{exo}[1][ ]{\ifthenelse{\equal{#1}{ }}{\proclaim{Exercise \arabic{exocomp}.}}{\proclaim{Exercise \arabic{exocomp} (#1).}}\stepcounter{exocomp}\hspace*{\fill}\linebreak\parindent=0pt}{\vskip 5mm}
%Fin des macros de mise en page.
%============================================================
\begin{document}
\Header{AstroStatistics Spring 2022 \\}{3 \\ \small {Issued: 30 March 2022 \hspace{0.9cm} Due: 14 April 2022} }


Useful Notes: 

\begin{itemize}
    \item \textbf{Unbiased estimators:} An estimator of a parameter $\theta$ is a function $T = T (X)$ which we use to estimate $\theta$ from an observation of $X$. $T$ is said to be unbiased if
    \begin{equation}
        \mathrm{E}(T) = \theta
    \end{equation}
    
    \item \textbf{Maximum likelihood estimation:} Suppose that the random variable $X$ has probability density function $f(x| θ)$. Given the observed value $x$ of $X$, the likelihood of $\theta$ is defined by
    \begin{equation}
        \mathcal{L}(\theta) = f(x|\theta)
    \end{equation}
    Thus we are considering the density as a function of $\theta$, for a fixed $x$. In the case of multiple observations we write the likelihood function as 
    \begin{equation}
        \mathcal{L}(\theta) = f(x_1, \dots, x_n | \theta) = \prod^{n}_{i=1}f(x_i| \theta)
    \end{equation}
    Thus the maximum likelihood estimate $\hat{\theta}(x)$ of $\theta$ is defined as the value of $\theta$ that maximizes the likelihood; we call it the maximum likelihood estimator (MLE) of $\theta$.\\
    
 
    \item \textbf{Sufficient statistics:} If MLE exists is always a function of the sufficient statistics. The sufficient statistics summarises all information which is relevant to do inference about $\theta$. It is important to know the following theorem called the factorization criterion:\\
    
    \textbf{Theorem: } \emph{The statistics $T$ is sufficient for $\theta$ if and only if $f(x | \theta)$ can be expressed as }
    \begin{equation}
        f(x | \theta) = g(T(X), \theta)h(x).
    \end{equation}
\end{itemize}

\vspace{0.5cm}
\section*{Problem 1}
Gaussian 1D problem. The surface temperature on Mars is measured by a probe 10 times, yielding the following data (units of K):
\begin{equation}
    \text{Data}: \{191.9,\quad 201.6,\quad 206.1,\quad 200.4,\quad 203.2,\quad 201.6,\quad 196.5,\quad 199.5,\quad 194.1,\quad 202.4\}
\end{equation}
\begin{enumerate}
    \item Assume that each measurement is independently Normally distributed with known variance $\sigma^2 = 25$ K2. What is the likelihood function for the whole data set?
    
    \item Find the Maximum Likelihood Estimate (MLE) for the surface temperature, $\hat{T}_{\text{ML}}$, and express your result to 4 significant figures accuracy.
    
    \item Determine symmetric confidence intervals at $68.3\%$, $95.4\%$, and $99\%$ around $\hat{T}_{\text{ML}}$ (4 significant figures accuracy)
    \item How many measurements would you need to make if you wanted to have a $1\sigma$ confidence interval around the mean of length less than 1 K (on each side)?
\end{enumerate}

\section*{Problem 2}
You flip a coin $n = 10$ times and you obtain $8$ heads.

\begin{enumerate}
    \item What is the likelihood function for this measurement? Identify explicitly what are the data and what is the free parameter you are trying to estimate.
    
    \item What is the Maximum Likelihood Estimate for the probability of obtaining heads in one flip, $p$?
    
    \item Approximate the likelihood function as a Gaussian around its peak and derive the $1\sigma$ confidence interval for $p$. How would you report your result for $p$?
    
    \item With how many $\sigma$ confidence can you exclude the hypothesis that the coin is fair? (\emph{Hint: compute the distance between the MLE for $p$ and $p = 1/2$ and
express the result in number of $\sigma$}).

    \item You now flip the coin $1000$ times and obtain $800$ heads. What is the MLE for $p$ now and what is the $1\sigma$ confidence interval for $p$? With how many $\sigma$ confidence can you exclude the hypothesis that the coin is fair now?
\end{enumerate}

\section*{Problem 3}
An experiment counting particles emitted by a radioactive decay measures $r$ particles per unit time interval. The counts are Poisson distributed.
\begin{enumerate}
    \item If $\lambda$ is the average number of counts per per unit time interval, write down the appropriate probability distribution function for $r$.
    
    \item Now we seek to determine $\lambda$ by repeatedly measuring for $M$ times the number
of counts per unit time interval. This series of measurements yields a sequence of counts $r = {\hat{r}_1, \hat{r}_2, \hat{r}_3, \dots, \hat{r}_M}$. Each measurement is assumed to be independent. Derive the joint likelihood function for $\lambda$, $\mathcal{L} (\lambda) = P(\hat{r}|\lambda)$, given the measured sequence of counts $\hat{r}$

\item Use the Maximum Likelihood Principle applied to the the log likelihood $\ln \mathcal{L} (\lambda)$ to show that the Maximum Likelihood estimator for the average rate
 $\lambda$ is just the average of the measured counts, $\hat{r}$, i.e
 \begin{equation}
     \hat{\lambda}_{\text{ML}} = \frac{1}{M}\sum_{i=1}^{M}\hat{r}_i
\end{equation}  
\item By considering the Taylor expansion of $\ln \mathcal{L} (\lambda)$ to second order around $\hat{\lambda}_{\text{ML}}$,
derive the Gaussian approximation for the likelihood $\mathcal{L} (\lambda)$ around the Maximum Likelihood point, and show that it can be written as
\begin{equation}
    \mathcal{L}(\lambda) \approx L_0\exp\Big(-\frac{1}{2}\frac{M}{\hat{\lambda}_{\text{ML}}}(\lambda -\hat{\lambda}_{\text{ML}})^2\Big)
\end{equation}
where $L_0$ is a normalisation constant.

\item Compare with the equivalent expression for $M$ Gaussian-distributed measurements to show that the variance $\sigma^2$ of the Poisson distribution is given by $\sigma^2 = \lambda$.
 
\end{enumerate}

\section*{Problem 4}
Suppose you measure the flux $F$ of photons from a laser source using 4 different instruments and you obtain the following results (units of 104 photons/cm$^2$):
\begin{equation}
    \text{Data: } \{34.7\pm5.0,\quad 28.9\pm2.0,\quad 27.1\pm3.0,\quad 30.6\pm4.0.\}
\end{equation}
\begin{enumerate}

\item Write down the likelihood for each measurement, and explain why a Gaussian approximation is justified in this case

\item Write down the joint likelihood for the combination of the 4 measurements.

\item Find the MLE of the photon flux, $\hat{F}_{\text{ML}}$, and show that it is given by:
\begin{equation}
    \hat{F}_{\text{ML}} = \sum_{i}\frac{\hat{n}_i}{\hat{\sigma}_i^2/\bar{\sigma}^2}
\end{equation}
where 
\begin{equation}
    \frac{1}{\bar{\sigma}^2}\equiv \sum_{i}\frac{1}{\hat{\sigma}^2_i}
\end{equation}
\item Compute $\hat{F}_{\text{ML}}$ from the data above and compare it with the sample mean.

\item Find the $1\sigma$ confidence interval for your MLE for the mean, and show that it is given by:

\begin{equation}
    \Bigg(\sum_i \frac{1}{\hat{\sigma}^2_i}\Bigg)^{-1/2}
\end{equation}
Evaluate the confidence interval for the above data. How would you summarize your measurement of the flux $F$?

\end{enumerate}

%\section*{Problem 5}
%Let the random variables drawn from the normal distribution: $X_1, X_2, \dots, X_n \sim N(\mu, \sigma^2)$. The parameters are $\theta = (\mu, \sigma^2)$, where $\mu$ and $\sigma$ are the mean and the standard deviation of the normal distribution to be estimated from the data. Using the factorization criterion discussed in the lecture, find the sufficient statistics for $\theta = (\mu, \sigma^2)$.

%Note that the sufficient statistics is not just a single number but can be a vector as in this case. This usually occurs when the parameter is a vector, $\theta = (\mu, \sigma^2)$. 

\section*{Problem 5} Suppose $X_1, X_2, \dots, X_n \sim N(\mu, \sigma^2)$, where $\mu$ and $\sigma^2$ are unknown and the be estimated from the data. 

\begin{enumerate}
    \item Find the maximum likelihood estimator (MLE) for $\mu$ and $\sigma$. The MLEs are defined as $\hat{\mu}_{\text{ML}}$ and $\hat{\sigma}^2_{\text{ML}}$
    
    \item Show that $\hat{\mu}_{\text{ML}}$ is unbiased but $\hat{\sigma}^2_{\text{ML}}$ is not unbiased. Find the unbiased estimator for the variance.
    
    \item Can you give an example of the estimator of the variance which is neither the MLE nor unbiased? (Hint: \emph{the estimator which minimizes the mean squared error. more Hint: estimator of the form $\lambda \sum^{n}_{i=1} (X_i - \bar{X})^2$}, for what value of $\lambda$ it is neither the MLE nor unbiased.)
\end{enumerate}

\section*{Problem 6}
In each of cases (a)–(c) write down the likelihood of $\theta$ and show that the stated $T(X)$ is a sufficient statistic for $\theta$.

In each case also find a MLE of $\theta$ and show that it is a function of $T(X)$. Find the distribution of $T(X)$ and determine whether or not the MLE is an unbiased estimator of $\theta$. If it is not, verify that it is asymptotically unbiased, and find some other estimator which is unbiased.
\begin{enumerate}
    \item $X_1, \dots, X_n$ are independent Poisson random variables, with $X_i$ having mean $i\theta$, where $\theta > 0$. $T(X) = \sum_{i=1}^n X_i$.
    
    \item $X_1, \dots, X_n$ are independent normal random variables, with $X_i \sim N(\theta, \sigma_i^2)$, and $\sigma^2_i, i=1, \dots, n$, known. $T(X) = \sum_{i=1}^n X_i/\sigma^2_i$.
    
    \item $X_1, \dots, X_n$ are $n>2$ independent and exponentially distributed random variables, with parameter $\theta$, i.e., with density $f(x | \theta) = \theta e^{-\theta x}, x > 0$. $T(X)=\sum_{i=1}^nX_i$
\end{enumerate}


\newline
Hint: \emph{In case (a), $T(X) \sim P(\frac{1}{2}n(n+1)\theta)$. In case (b), $T(X) \sim N(\theta \sum_i \sigma_i^{-2}, \sum_i \sigma_i^{-2})$. In case (c), $T(X) \sim gamma(n,\theta)$. Do you understand why? }

\end{document}
